\documentclass[11pt, a4paper]{article}
%\usepackage{geometry}
\usepackage[inner=1.5cm,outer=1.5cm,top=2.5cm,bottom=2.5cm]{geometry}
\pagestyle{empty}
\usepackage{graphicx}
\usepackage{fancyhdr, lastpage, bbding, pmboxdraw}
\usepackage[usenames,dvipsnames]{color}
\definecolor{darkblue}{rgb}{0,0,.6}
\definecolor{darkred}{rgb}{.7,0,0}
\definecolor{darkgreen}{rgb}{0,.6,0}
\definecolor{red}{rgb}{.98,0,0}
\usepackage[colorlinks,pagebackref,pdfusetitle,urlcolor=darkblue,citecolor=darkblue,linkcolor=darkred,bookmarksnumbered,plainpages=false]{hyperref}
\renewcommand{\thefootnote}{\fnsymbol{footnote}}

\pagestyle{fancyplain}
\fancyhf{}
\lhead{ \fancyplain{}{Course Name} }
%\chead{ \fancyplain{}{} }
\rhead{ \fancyplain{}{\today} }
%\rfoot{\fancyplain{}{page \thepage\ of \pageref{LastPage}}}
\fancyfoot[RO, LE] {page \thepage\ of \pageref{LastPage} }
\thispagestyle{plain}

%%%%%%%%%%%% LISTING %%%
\usepackage{listings}
\usepackage{caption}
\DeclareCaptionFont{white}{\color{white}}
\DeclareCaptionFormat{listing}{\colorbox{gray}{\parbox{\textwidth}{#1#2#3}}}
\captionsetup[lstlisting]{format=listing,labelfont=white,textfont=white}
\usepackage{verbatim} % used to display code
\usepackage{fancyvrb}
\usepackage{acronym}
\usepackage{amsthm}
\VerbatimFootnotes % Required, otherwise verbatim does not work in footnotes!



\definecolor{OliveGreen}{cmyk}{0.64,0,0.95,0.40}
\definecolor{CadetBlue}{cmyk}{0.62,0.57,0.23,0}
\definecolor{lightlightgray}{gray}{0.93}



\lstset{
%language=bash,                          % Code langugage
basicstyle=\ttfamily,                   % Code font, Examples: \footnotesize, \ttfamily
keywordstyle=\color{OliveGreen},        % Keywords font ('*' = uppercase)
commentstyle=\color{gray},              % Comments font
numbers=left,                           % Line nums position
numberstyle=\tiny,                      % Line-numbers fonts
stepnumber=1,                           % Step between two line-numbers
numbersep=5pt,                          % How far are line-numbers from code
backgroundcolor=\color{lightlightgray}, % Choose background color
frame=none,                             % A frame around the code
tabsize=2,                              % Default tab size
captionpos=t,                           % Caption-position = bottom
breaklines=true,                        % Automatic line breaking?
breakatwhitespace=false,                % Automatic breaks only at whitespace?
showspaces=false,                       % Dont make spaces visible
showtabs=false,                         % Dont make tabls visible
columns=flexible,                       % Column format
morekeywords={__global__, __device__},  % CUDA specific keywords
}

%%%%%%%%%%%%%%%%%%%%%%%%%%%%%%%%%%%%
\begin{document}
\begin{center}
{\Large \textsc{Programmation avancée en Python (PAPY)}}
\end{center}
\begin{center}
2022-2023
\end{center}
%\date{September 26, 2014}

\begin{center}
\rule{6in}{0.4pt}
\begin{minipage}[t]{.75\textwidth}
\begin{tabular}{llcccll}
\textbf{Enseignant:} & Lucas Lestandi & & &  & \textbf{Volume:} & 32h \\ %\textbf{Time:} & F 14:00 -- 17:00
\textbf{Email:} &  \href{mailto:XYZ@email.org}{lucas.lestandi@ec-nantes.fr} & & & & \textbf{Bureau:} & T123
\end{tabular}
\end{minipage}
\rule{6in}{0.4pt}
\end{center}
\vspace{.5cm}
\setlength{\unitlength}{1in}
\renewcommand{\arraystretch}{2}

% \noindent\textbf{Course Pages:} \begin{enumerate}
% \item \url{hypocampus}
% \end{enumerate}
%
\noindent\textbf{Répartion horaire:}
\begin{enumerate}
\item 8h CM
\item 22h TP
\item 2h DS
\end{enumerate}
% \vskip.15in
% \noindent\textbf{Office Hours:} After class, or by appointment, or post your questions in the forum provided for this purpose on AeLP.

\vskip.15in
% \noindent\textbf{Main References:} %\footnotemark
% TODO
% \begin{itemize}
% % \item  xx, {\textit{Introduction a python}}
% % \item \url{https://www.w3schools.com/python/python_intro.asp}
% \end{itemize}

% \footnotetext{Downloadable ebook versions are available on AeLP.}
\vskip.15in
\noindent\textbf{Mots-clés:} environnement, objets, introspection, modules, architecture, packaging

\vskip.15in
\noindent\textbf{Objectifs:}
À la fin de ce cours, les étudiants et étudiantes sauront :
\begin{enumerate}
  \item Mettre en place un environnement python robuste ainsi qu'un environnement de développement adapté.
  \item Écrire des programmes python efficaces utilisant les bibliothèques et abstractions modernes.
  \item Assurer la lisibilité et la maintenance en suivant les principes de structuration modulaire et de style cohérents avec celles de la communauté.
  \item Créer un paquet pour la distribution sur d'autres machines y compris des dépendances.
\end{enumerate}


\vskip.15in
\noindent\textbf{Plan du cours:}
\begin{enumerate}
  \item Bien travailler avec python
  \begin{itemize}
    \item choisir un environnement de travail: IDE, notebooks jupyter, environnement python (conda,...)
    \item philosophie et syntaxe de python
    \item l'interpréteur python
    \item bonnes pratiques en programmation python (PEP8,...)
  \end{itemize}
  \item Structuration et types de données
  \begin{itemize}
    \item Variables, références et gestion de la mémoire
    \item Types de données et structures
    \item Programmation orientée objet (OOP) : les classes
    \item Écrire du code robuste : architecture, introspection, exceptions, etc.
  \end{itemize}
  \item Programmer avec des modules
  \begin{itemize}
    \item les modules natifs : os, sys, subprocess,...
    \item les bibliothèques externes : avec pip ou conda
    \item quelques exemples communs : numpy, scipy, matplotlib
    \item créer ses propres modules
  \end{itemize}
  \item Pour aller plus loin
  \begin{itemize}
    \item les décorateurs
    \item communication avec d'autres langages (C++,...)
    \item tester et debugger
    \item distribution and portabilité des paquets
  \end{itemize}
\end{enumerate}


\vspace*{.15in}
\noindent\textbf{Modalités d'évaluation:}
\begin{itemize}
  \item TPs notés (coefficient 1/2)
  \item DS (coefficient 1/2)
\end{itemize}

\section*{English}
\noindent\textbf{Keywords:} environment, OOP, introspection, modules, architecture, packaging


\noindent\textbf{Objectives:}
At the end of this course, student will have the skills to
\begin{enumerate}
  \item Set up a robust python environment and a suitable development environment.
  \item Write efficient python programs using libraries and modern abstractions.
  \item Ensure readability and maintenance by following modular architecture and community style guidelines.
  \item Package their code for distribution on other machines including dependencies.
\end{enumerate}

\vskip.15in
\noindent\textbf{Course outline:}
\begin{enumerate}
  \item The right way to work with python
  \begin{itemize}
    \item setting up the right environment : IDE, jupyter notebooks, python environments (conda,...)
    \item language philosophy, syntax
    \item python interpreter
    \item good practices for programming in python (PEP8 style guidelines,...)
  \end{itemize}
  \item Code structure and data types
  \begin{itemize}
      \item variables, memory and references
      \item data types and structures
      \item object oriented programming (OOP) : classes
      \item writing robust code : architecture, introspection, exceptions,...
  \end{itemize}
  \item Using modules
  \begin{itemize}
    \item native modules : os, sys, subprocess
    \item external libraries : using pip and conda
    \item useful examples : numpy, scipy, matplotlib
    \item create your own modules
  \end{itemize}
  \item Towards production code
  \begin{itemize}
    \item decorators
    \item integration with other langages (C++,...)
    \item testing and debugger
    \item distribution and portability
  \end{itemize}
\end{enumerate}

\vspace*{.15in}
\noindent\textbf{Grading Policy:}
\begin{itemize}
  \item Labs report (1/2)
  \item Final exam (1/2)
\end{itemize}
%
% \begin{itemize}
%   \item 2 jupyter notebook in report style
%   \item 1 program that can be installed and run on a clean environment. Subject to be determined
% \end{itemize}

% \vskip.15in
% \noindent\textbf{Important Dates:}
% \begin{center} \begin{minipage}{3.8in}
% \begin{flushleft}
% Midterm \#1      \dotfill ~\={A}b\={a}n 16, 1393  \\
% Midterm \#2      \dotfill ~\={A}zar 21, 1393  \\
% %Project Deadline \dotfill ~Month Day \\
% Final Exam       \dotfill ~Dey 18, 1393  \\
% \end{flushleft}
% \end{minipage}
% \end{center}
%
% \vskip.15in
% \noindent\textbf{Course Policy:}
% \begin{itemize}
% \item Please sign up for AeLP. I will confirm your enrollment for the course, then you will be able to see the course page.
%
% \end{itemize}
%
% \vskip.15in
% \noindent\textbf{Class Policy:}
% \begin{itemize}
% \item Regular attendance is essential and expected.
% \end{itemize}
%
% \vskip.15in
% \noindent\textbf{Academic Honesty:}   Lack of knowledge of the academic honesty policy is not a reasonable explanation for a violation.


%%%%%% THE END
\end{document}
